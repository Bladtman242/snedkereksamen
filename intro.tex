\section*{Introduktion}
Formålet dette projekt er at illustrere relevante færdigheder, tilegnet på
snedkeruddannelsens grundforløb, KTS.
Projektet består af denne rapport samt billag, og det møbel rapporten omhandler.
Projektet omfatter blandt andet; design og arbejdstegning i CAD program
(\texttt{SolidWorks}), procesbeskrivelse, skæresedel, bestilling af råtræ uden
overdreven spild, behandling af råtræ, maskin- og hånd-lavede samlinger og finerarbejde.

Projektet er et natbord, inspireret af to lignende projekter fra
\texttt{CITYJOINERY}\nolinebreak \footnote{\texttt{cityjoinery.com}}.

\begin{figure}[htb]
\centering
\fbox{
\includegraphics[width=0.4 \textwidth]{imgs/Aspiration-Nightstand-birch.jpg}
\includegraphics[width=0.4 \textwidth]{imgs/Aspiration-Nightstand-walnut.jpg}
}
\caption{Ende- og nat-bord fra \texttt{CITYJOINERY} i hhv. birk og valnød til
venstre, og valnød til højre.}
\end{figure}

\section*{Koncept \& Design}
Projektet er et fribensmøbel med sarge, én skuffe med hylde over,
og en bordplade. Ben, sarge og skuffe front udføres i massiv valnød,
skuffebund, hylde og bordplade, laves af birke-krydsfiner med
valnøddefiner og kantlister, skuffesiderne udføres i elm. Skuffen styres af
flere lister bag sargene, gjort i ahorn.

\subsection*{Materialer}
Natbordets bordplade og hylde udføres i krydsfiner, da det er mere stabilt
end massivtræ. Da begge elementer fæstnes til benene, kan de potentielt
skævvride hele møblet, hvis de skulle slå sig med tiden. Brugen af finerede
pladematerialer løser det problem. Alternativt kunne man have brugt samlinger
der giver emnerne mulighed for at forskyde sig i forhold til hinanden, uden
større påvirkning på resten af møblet.

Valget af elm til skuffesiderne er kontroversielt. Traditionelt bruger man gerne
ahorn til skuffesider.
Elm er kendt for at være splintret, med et ujævnt fiber forløb, der kan virke
sløvende på værktøjer, og let får oprifter\footnotemark. I dette projekt bruges
elm primært for sin æstetiske værdi, da kerneveddet kan have en række
forskellige farver (lilla og grønne strøg langs årene). Denne æstetiske
finurlighed vælges her fremfor den lettere arbejdsproces og større farve
kontrast ahorn havde givet (kontrast til skuffefronten i amerikansk valnød). Det
viser sig imidlertid at elm har flere glimrende egenskaber. Selvom det har
omtrent samme volumensvind ved tørring som ahorn ($\sim12\%$, ahorn $\sim11\%$)
er svindet tangentialt mindre, relativt til svindet radialt. Ahorn har, ligesom
egetræ, et tangentialsvind der er mere end dobbelt så stort som radialsvindet.
Uligheden mellem tangential- og radial-svind er den primære årsag til at træ
skælvrides når fugtigheden varieres. Hvis det radiale, tangentiale og aksiale
svind var det samme, ville træet blot variere i størrelse med fugtigheden, uden
at skævvrides. Som et kuriosum kan det nævnes at indersiden af elmebark, i tider
med hungersnød, har været brugt som erstatning for mel i bagværk.

\footnotetext{Når andet ikke er angivet, er fakta om træsorter taget fra
\texttt{Træarter - TRÆ69}, fra \texttt{Træinformation}. ISBN 978-87-90856-328}

\subsection*{Udskæringer}
Natbordet har fået flere udskæringer og taperinger, der skal få det til at
fremstå mindre og lettere. Hvert ben er taperet på begge ydersider,
hvilket  efterlader indersiderne -- ind mod resten af
møblet -- som lodrette plan der bruges som reference for skuffen og sargene.
Modsat taperingerne er toppen af benene bortfræset fra møblets inderside, hvilket
giver benenes taperede ydersider et øget blikfang. Det giver illusionen af at
benene er montereret skrånende på møblet, selvom indersiderne
er lodrette. Det ønskes at sargenes yderside i bunden flugter benenes, ligesom
deres indersider er koplanare med benenes indersider. Førstnævnte for visuel
sammenhæng, sidstnævnte for at skuffen kan styres af sargene, uden brug af
styrelister. For at det kan opnås, er sargene 34mm tykke, svarende til benenes
tykkelse, målt 200mm fra toppen. [*ILLUSTRATION*]

[*HØVL SKUFFE BAGSIDE*]
For at mindske friktion, og risiko for skader, bør åreretningen på skuffens
bærelister gå langs skuffesiderne når skuffen skubbes ind. Det samme gælder
åreretningingen på skuffesiderne. Dette opnås nemt på bærelisterne i møblets
sider, men er mere problematisk for den bagerste og forreste bæreliste.
Problemet løses ved at høvle en smule af undersiden af skuffens bagside, så den
ikke kommer i kontakt i med bagerste bæreliste, og ligeledes ved at tage en
smule at begge ender på forreste bæreliste, hvis hovedformål alligevel er at
stabilisere de forreste ben, der ikke har en sarg imellem sig, samt at fungere
som et ekstra dybdestop for skuffen, så noget af slaget tages af den
bagerste sarg når skuffen lukkes.
